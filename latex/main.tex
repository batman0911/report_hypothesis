% 	This template is  MIT licensed.

% 	Basic file to demonstrate the usage of this LaTeX template.
% 	You can build your own paper/thesis on top of this file.
% 	Simply adjust the document class and all metadata and start working.
%
\documentclass[
	language=english, % set to english or german
	type=master, % set to bachelor, master or seminar
]{isthesis}

\usepackage[utf8]{vietnam}

% custom package
\usepackage{eurosym}
\usepackage{makecell}

% Graphics rendering using TikZ
% See: https://en.wikibooks.org/wiki/LaTeX/PGF/TikZ
\usepackage{tikz}
% Include required TikZ libraries here, some exemplary libraries are pre-included
\usetikzlibrary{calc}
\usetikzlibrary{matrix}
\usetikzlibrary{positioning}
\usetikzlibrary{shapes.geometric}

%Add your library here
\addbibresource{library.bib}

% Import acronyms
% \newacronym[longplural={<long plural>}, shortplural={<short plural>}]{<label>}{<short>}{<long>}
% 	label = is the unique identifier and sort key for the acronym, can be the same as <short>
%	short = is the abbreviation or acronym
%	short plural (optional) = is the plural of the abbreviation or acronym
%	long = is the long form of the acronym, this will appear in the list of abbreviations
%	long plural (optional) = is the long plural form of the abbreviation or acronym

\newacronym[shortplural={KMUen}, longplural={Kleine und Mittlere Unternehmen}]{kmu}{KMU}{Kleines und Mittleres Unternehmen}
\newacronym{CD}{CD}{Corporate Design}
\newacronym{SQL}{SQL}{Structured Query Language}
\newacronym{FAU}{FAU}{Khoa Toán cơ tin}
\newacronym{BPM}{BPM}{Business Process Management}
\newacronym{npm}{NPM}{Node Package Manager}
\newacronym{diss}{DISS}{Digital Industrial Service System}

% Import symbols
% Syntax: <Symbol> <Label> <Name>
% The symbols are sorted by their labels
\addsymboltolist{$\Pi$}{Pi}{Projection}
\addsymboltolist{$\Join$}{Join}{Natural Join}
\addsymboltolist{$\sigma$}{Selection}{Selection}


% Import custom commands
% If you want to define custom commands, please do so here

% Import custom code block
% define listing code
\definecolor{codegreen}{rgb}{0,0.6,0}
\definecolor{codegray}{rgb}{0.5,0.5,0.5}
\definecolor{codepurple}{rgb}{0.58,0,0.82}
\definecolor{backcolour}{rgb}{0.95,0.95,0.92}

\lstdefinestyle{mystyle}{
    backgroundcolor=\color{backcolour},   
    commentstyle=\color{codegreen},
    keywordstyle=\color{magenta},
    numberstyle=\tiny\color{codegray},
    stringstyle=\color{codepurple},
    basicstyle=\ttfamily\footnotesize,
    breakatwhitespace=false,         
    breaklines=true,                 
    captionpos=b,                    
    keepspaces=true,                 
    numbers=left,
    firstnumber=1,
    stepnumber=1,                    
    numbersep=5pt,                  
    showspaces=false,                
    showstringspaces=false,
    showtabs=false,                  
    tabsize=2,
    framesep=10pt,
    xleftmargin=10pt,
    xrightmargin=10pt,
    framexleftmargin=16pt,
    framextopmargin=2pt,
    framexbottommargin=2pt, 
    frame=tb, framerule=0pt,
}

\lstset{style=mystyle}

% Document meta information
\isthesis{
    title={TIỂU LUẬN},
    sub-title={NHẬP MÔN SUY DIỄN THỐNG KÊ \\ 
    Kiểm định giả thiết một biến số},
    author-name={NGUYỄN MẠNH LINH}, % Separate multiple authors with commas
    % author-email={linhnguyen.code@gmail.com},
    % author-matriculation={MATRICULATION NUMBER},
    % author-phone={+49 XXXXXXXXX}, % Use international numbers format
    % author-address={STREET},
    % author-zip={ZIP},
    % author-city={CITY},
    principal-supervisor={Hoàng Phương Thảo}, % This must be a professor
    % associate-supervisor={SUPERVISOR}, % This is your main supervisor, i.e., a post doc or doctoral student
    tutor-supervisor={}, % If required, define an additional supervisor resp. tutor here
    group-institute={Đại học Khoa học Tự nhiên},
    % group={Image Data Exploration and Analysis (IDEA) Lab},
    % studies={M.Sc. International Information Systems}, %your field of studies, i.e. Wirtschaftsinformatik or International Information Systems
    %
    %associate-group={}, % When the thesis is done in cooperation with another chair, add it here
    %associate-group-institute={}, % add cooperating institute or university here
    seminar={SEMINAR}, % The title of your seminar
    submission-date={2022-01-01} % The date you handed in your document: Format yyyy-mm-dd
    %primary-logo={}, % Uses the FAU logo by default
    %primary-logo-height={}, % Uses 16mm as default height
    %secondary-logo={}, % Logo of the secondary institution (cooperating chair/university), USES Faculty logo by default
    %secondary-logo-height={} % Uses 16mm as default height
}


\begin{document}
    % Title page
    \newcounter{savepage}
    \maketitle

	% Quote
    % You can put an optional quote page in front of your content
    %   \quotepage[author={Arthur C. Clarke}]{
    %   	        Any sufficiently advanced technology is indistinguishable from magic.
    %   }
    \begin{abstract}
	    % Add your abstract here:
        Một kiểm định giả thiết trong mô hình Neyman-Pearson là một tiêu chí quyết định 
        cho phép chúng ta lựa chọn giữa 2 giả thiết. Trước khi thực hiện test thống kê, 
        ta định nghĩa giả thiết \textbf{null} $H_0$, được giả định là đúng. Giả thiết được
        so sánh với đối thiết $H_1$. Đối thiết $H_1$ thường được gọi là giả thiết nghiên cứu
        vì thường về lí thuyết các tham số được chỉ định trong giải thiết thay thế này. \\
        Bài này sẽ nghiên cứu những kiến thức cơ bản về kiểm định giả thiết và các phương pháp
        kiểm định cho trung bình, độ lệch và tỉ lệ.
        enter something here, somthing here more
		% \lipsum[1]
	\end{abstract}
    
    % Table of contents
    \tableofcontents

    % List of figures (if you have figures)
    % \listoffigures

    % List of tables (if you have tables)
    % \listoftables
    
    % List of listings (if you have listings)
	% \lstlistoflistings

    % List of abbreviations (if you use acronyms)
    %\listofabbreviations

    % List of symbols (if you use symbols)
    %\listofsymbols
	
	% Abstract
	%
	% Comment out this part, if you don't require an abstract

	
	% storing the last pagenumber
    \setcounter{savepage}{\value{page}}
    
    
    % Content
    \begin{content}
        % Add your content files:
		\chapter{Giới thiệu}
Một kiểm định giả thiết trong mô hình Neyman-Pearson là một tiêu chí quyết định 
cho phép chúng ta lựa chọn giữa 2 giả thiết. Trước khi thực hiện test thống kê, 
ta định nghĩa giả thiết \textbf{null} $H_0$, được giả định là đúng. Giả thiết được
so sánh với đối thiết $H_1$. Đối thiết $H_1$ thường được gọi là giả thiết nghiên cứu
vì thường về lí thuyết các tham số được chỉ định trong giải thiết thay thế này. \\

Các giả thiết có 1 miền xác định tham số  trong không gian tham số $\Theta$ của 
các tham số $\theta$. Giả thiết null $H_0$ được định nghĩa trong miền 
$[\theta \in \Theta_0]$ và đối thiết $H_1$ được định nghĩa trong miền
$[\theta \in \Theta_1]$ và $\Theta_0 \cup \Theta_1 = \Theta$

\paragraph{TODO}
\begin{itemize}
	\item Configuration switch for having \textbackslash chapter\{\} begin on a new page
	\item Replace \texttt{kvoptions} with \texttt{pgfkeys}
\end{itemize}
		\chapter{Kiểm định giả thiết cho trung bình}

\section{Kiểm định giả thiết cho trung bình với mẫu có phân bố chuẩn và phương sai đã biết}
This template provides three structural levels that appear in the table of contents: \texttt{\textbackslash chapter}, \texttt{\textbackslash section}, and \texttt{\textbackslash subsection}. Chapters will always start on a new page. Additionally, you can use \texttt{\textbackslash subsubsection} and \texttt{\textbackslash paragraph} as non-hierarchical means to structure your thesis.

demo code: 
\begin{lstlisting}[]
    sigma <- 6
    mu <- 40
    pnorm(-2)
    pnorm(2)
\end{lstlisting}



\subsection{Lists}
You can use the default \LaTeX \- functions for writing lists, viz., \texttt{\textbackslash enumerate} for numbered lists and \texttt{\textbackslash itemize} for bullet point lists. Again, the \texttt{\textbackslash subsubsection} and \texttt{\textbackslash paragraph} can be used as structural elements, e.g., when listing definitions of terms.



		\chapter{Kiểm định cho phương sai}

\section{Kiểm định cho phương sai với mẫu có phân bố chuẩn}

Cho $n$ mẫu ngẫu nhiên $X_1, X_2, .. , X_n$ có phân bố chuẩn $N(\mu, \sigma)$, 
đại lượng chuẩn hóa sau có phân bố $\chi^2$:
\begin{equation}
    \frac{(n - 1)S^2}{\sigma^2} \sim \chi^2_{n - 1}
\end{equation}
Giả thiết null cho phương sai $H_0: \sigma^2 = \sigma_0^2$, giá trị test thống kê:
$$\chi_{obs}^2 = \frac{(n - 1)S^2}{\sigma_0^2}$$
Ba đối thiết và miền bác bỏ $H_0$ tương ứng cho dưới bảng sau:

\begin{center}
    \begin{tabular}{| c | c | c | c |}
        \hline
        Đối thiết & $H_1: \sigma^2 < \sigma_0^2$ & $H_1: \sigma^2 > \sigma_0^2$ & $H_1: \sigma^2 \neq \sigma_0^2$ \\
        Miền bác bỏ & $\chi_{obs}^2 < \chi_{\alpha, n - 1}^2$ & $\chi_{obs}^2 > \chi_{\alpha, n - 1}^2$ & $\chi_{obs}^2 < \chi_{\alpha/2, n - 1}^2 \cup \chi_{obs}^2 > \chi_{1 - \alpha/2, n - 1}^2$ \\
        \hline
    \end{tabular}
\end{center}
        \chapter{Kiểm định giả thiết cho tỉ lệ}

\section{Kiểm định cho tỉ lệ với trong test nhị thức}

Ta xem xét biến ngẫu nhiên có phân bố nhị thức $Y \sim Bin(n, \pi)$, giả thiết null
$H_0: \pi = \pi_0$. 
Test thống kê cho $y_{obs} = \text{Số quan sát thành công}$

Ta có 3 đối thiết ứng với công thức p-value tương ứng như sau:

\begin{center}
    \begin{tabular}{| c | c |}
        \hline
        Đối thiết & p-value \\
        \hline
        $H_1: \pi < \pi_0$ & $P(Y \leqslant y_{obs} | H_0) = \sum_{i = 0}^{y_{obs}}\begin{pmatrix}
            n \\ i
        \end{pmatrix} \pi_0^i(1 - \pi_0)^{n - i}$ \\ 
        \hline 
        $H_1: \pi > \pi_0$ & $P(Y \geqslant y_{obs} | H_0) = \sum_{i = y_{obs}}^{n}\begin{pmatrix}
            n \\ i
        \end{pmatrix} \pi_0^i(1 - \pi_0)^{n - i}$ \\
        \hline
        $H_1: \pi \neq \pi_0$ & $\sum_{i = 0}^n I(P(Y = i) \leqslant P(Y = y_{obs}))\begin{pmatrix}
            n \\ i
        \end{pmatrix} \pi_0^i (1 - \pi_0)^{n- i}$ \\
        \hline
    \end{tabular}
\end{center}

\section{Kiểm định cho tỉ lệ trong test nhị thức (xấp xỉ chuẩn)}
Theo tính chất tiệm cận của ước lượng MLE ta có:

\begin{equation}
    P = \frac{Y}{n} \sim N\left( \pi, \sqrt{\frac{\pi(1 - \pi)}{n}} \right) \text{với } n \to \infty
\end{equation}

Trong thực tế Với $n\pi$ và $n(1 - \pi)$ đều lớn hơn hoặc bằng 10, ta có thể sử dụng xấp xỉ trên.

Đại lượng chuẩn hóa:
\begin{equation}
    Z = \frac{P - \pi_0}{\sqrt{\frac{\pi_0(1 - \pi_0)}{n}}} \sim N(0, 1)
\end{equation}

Khi $|p - \pi_0| > \frac{1}{2n}$ nhiều nhà thống kê cho rằng cần phải thêm vào 1 hiệu chỉnh. 
Hàm \lstinline{prop.test()} trong \lstinline{R} mặc định thêm vào hiệu chỉnh $\pm \frac{1}{2n}$.

Ta có bảng tổng quan như sau

Giải thiết null $H_0: \pi = \pi_0$

\begin{center}
    \begin{tabular}{| c | c |}
        \hline
        Đối thiết & miền bác bỏ \\
        \hline
        $H_1: \pi < \pi_0$ & $z_{obs} < z_\alpha$ \\ 
        \hline 
        $H_1: \pi > \pi_0$ & $z_{obs} > z_{1 - \alpha}$ \\
        \hline
        $H_1: \pi \neq \pi_0$ & $|z_{obs}| > z_{1 - \alpha / 2}$ \\
        \hline
    \end{tabular}
\end{center}

Đối với $|p - \pi_0| > \frac{1}{2n}$ ta có bảng hiệu chỉnh như sau

\begin{center}
    \begin{tabular}{| c | c | c |}
        \hline
        Điều kiện & Hiểu chỉnh & Tham số chuẩn hóa\\
        \hline
        $p - \pi_0 > 0$ & $-\frac{1}{2n}$ & $z_{obs} = \frac{p - \pi_0 - 
        \frac{1}{2n}}{\sqrt{\frac{\pi_0(1 - \pi_0)}{n}}}$ \\ 
        \hline 
        $p - \pi_0 < 0$ & $\frac{1}{2n}$ & $z_{obs} = \frac{p - \pi_0 + 
        \frac{1}{2n}}{\sqrt{\frac{\pi_0(1 - \pi_0)}{n}}}$ \\ 
        \hline
    \end{tabular}
\end{center}

\section{Bài tập}

\textbf{pr17}: Theo Pamplona, Tây Ban Nha, $0.4\%$ người nhập cư năm 2002 đến từ Bolivia.
Tháng 6 năm 2005, một nhóm gồm 3740 người nước ngoài được điều tra ngẫu nhiên thì có 87 người 
Bolivia. Có thể phòng đoán số người nhập cư Boliva đã tăng lên hay không với mức ý nghĩa $\alpha = 0.05$.

Giả thiết null: $H_0: P = \pi_0 = 0.004$, đối thiết $P > \pi_0 = 0.004$
Với $n = 3740$, ta có $n\pi_0$ và $n(1 - \pi_0)$ đều lớn hơn 10 nên ta có thể sử dụng xấp xỉ chuẩn 
để kiểm định cho tỉ lệ trong bài toán này.

Ta có $P = 87/3740$ nên $P - \pi_0 = 0.01926203 > 1/2n = 0.0001336898$, do đó ta phải sử dụng hiệu chỉnh $1/2n$

tham số chuẩn hóa 
$$Z = \frac{P - \pi_0 - \frac{1}{2n}}{\sqrt{\frac{\pi_0(1 - \pi_0)}{n}}} = 18.53333 > 1.644854 = Z_{0.95}$$

Do đó ta có thể kết bác bỏ giả thiết $H_0$

Kết luận: với mức ý nghĩa $5\%$ ta có thể nói rằng tỉ lệ người nhập cư vào Tây Ban Nha đến từ Bolivia (điều tra tháng 6 năm 2005)
đã tăng lên so với thời điểm năm 2002.

Ngoài ra, chúng ta có thể sử dụng phương pháp tính giá trị \lstinline{p-value} được đề cập trong phần 4.1.
Hoặc có thể sử dụng test dựng sẵn trong \lstinline{R} đơn giản như sau:

\begin{lstlisting}
    binom.test(x = 87, n = 3740, p = 0.004, alternative = "greater")
\end{lstlisting}

Kết quả thu được:

\begin{lstlisting}
        Exact binomial test

    data:  87 and 3740
    number of successes = 87, number of trials = 3740, p-value < 2.2e-16
    alternative hypothesis: true probability of success is greater than 0.004
    95 percent confidence interval:
    0.01935316 1.00000000
    sample estimates:
    probability of success 
                0.02326203 
\end{lstlisting}

$\text{\lstinline{p-value}} < \alpha = 0.05$, ta có kết luận tương tự như trên.
    \end{content}
    
    \pagenumbering{Roman}
    \setcounter{page}{\numexpr\value{savepage}}

    % References
    % \references{}
    
    % Appendix
     \begin{appendix}
        % In the appendices, use \section{} instead of \chapter{}
         \section{Some Appendix Section}
\label{sec:appendix01}
Appendices provide only two structural levels, viz., \texttt{\textbackslash section}, and \texttt{\textbackslash subsection}.

The numbering of figures, listings, tables, and footnotes is not reset. Thus, it continues as usual in the appendix.

\subsection{Some Appendix Subsection}

\lipsum[10]
     \end{appendix}




    % Declaration of authorship
    % \authorshipstatement[pagenumbering=false]
    \authorshipstatement[pagenumbering=true]
    % \authorshipstatement[pagenumbering=only]
    
    % Consent form for use of plagiarism detection software
    % Not yet required
    % \consentform[pagenumbering=false]
    % \consentform[pagenumbering=true]
    % \consentform[pagenumbering=only]
    
    % Bonus: Wordcount
    % cd %FOLDER WHERE THE .tex FILES ARE IN %
    % clear
    % texcount -total -q -col -sum *.tex
    
\end{document}