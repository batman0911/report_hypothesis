\chapter{Giới thiệu}
Một kiểm định giả thiết trong mô hình Neyman-Pearson là một tiêu chí quyết định 
cho phép chúng ta lựa chọn giữa 2 giả thiết. Trước khi thực hiện test thống kê, 
ta định nghĩa giả thiết \textbf{null} $H_0$, được giả định là đúng. Giả thiết được
so sánh với đối thiết $H_1$. Đối thiết $H_1$ thường được gọi là giả thiết nghiên cứu
vì thường về lí thuyết các tham số được chỉ định trong giải thiết thay thế này. \\

Các giả thiết có 1 miền xác định tham số  trong không gian tham số $\Theta$ của 
các tham số $\theta$. Giả thiết null $H_0$ được định nghĩa trong miền 
$[\theta \in \Theta_0]$ và đối thiết $H_1$ được định nghĩa trong miền
$[\theta \in \Theta_1]$ và $\Theta_0 \cup \Theta_1 = \Theta$

\section{Sai lầm loại I và sai lầm loại II}
Khi sử dụng phương pháp kiểm định giả thiết chúng ta luôn có xác suất mắc sai lầm.
Ví dụ chúng ta có thể bác bỏ giả thiết null trong khi giả thiết trên thực tế là đúng
hoặc ngược lại, giả thiết null trên thực tế là sai mà chúng ta lại chấp nhận nó. \\ 
Ta định nghĩa 2 loại sai lầm và sẽ tìm hiểu xác suất mắc phải sai lầm đó trong kiểm định 
giả thiết thống kê.

\begin{center}
    \begin{tabular}{| c | c |}
        \hline
        Sai lầm loại I & Giả thiết null là đúng nhưng bị bác bỏ \\
		Sai lầm loại II & Giả thiết null là sai nhưng không bị bác bỏ \\
        \hline
    \end{tabular}
\end{center}

Cả 2 loại này đều dẫn đến quyết định sai lầm trong kết luận.\\
Xác suất mắc sai lầm loại I còn được gọi là \textbf{mức ý nghĩa}
\begin{equation}
	\begin{split}
		\alpha = P(\text{sai lầm loại I}) = P(\text{bác bỏ } H_0 | H_0 \text{ đúng}) \\
		= P(\text{chấp nhận } H_1 | H_0 \text{ đúng})	
	\end{split}
\end{equation}
Xác suất mắc sai lầm loại II $\beta$:
\begin{equation}
	\begin{split}
		\beta = P(\text{sai lầm loại II}) = P(\text{không bác bỏ } H_0 | H_0 \text{ sai}) \\
	= P(\text{chấp nhận } H_0 | H_1 \text{ đúng})
	\end{split}
\end{equation}

\section{Power function}
Đối thiết $H_1: \theta \in \Theta_1$, hàm power của kiểm định được định nghĩa như sau:
\begin{equation}
	\begin{split}
		Power(\theta) = P(\text{bác bỏ } H_0 | H_0 \text{ sai}) = P(\text{chấp nhận } H_1 | H_1 \text{ đúng}) \\
		= 1 - \beta(\theta)
	\end{split}
\end{equation}
Với $\beta(\theta)$ là xác suất mắc sai lầm loại II với 1 ước lượng $\theta$ cho trước.

\section{Kiểm định đồng nhất tốt nhất}
Trước hết chúng ta cùng xem xét ví dụ sau đây \\
\textbf{Ví dụ: } Cho một tổng thể với phân bố chuẩn $N(\mu, 1)$, thực hiện 1 phép thử. 
Kiểm định giả thiết $H_0: \mu = 1$ với đối thiết $H_1: \mu = 2$. Tìm mức ý nghĩa và power của kiểm định 
trong các miền bác bỏ sau: \\
(a) $(2.036, \infty)$ \\
(b) $(1.100, 1.300) \cup (2.461, \infty)$

(a) $R = (2.036, \infty),
$$\alpha = P(X > 2.036 | N(1, 1)) = P(\frac{X-1}{1} > \frac{2.036-1}{1}) = P(Z > 1.036) = 0.150$$
