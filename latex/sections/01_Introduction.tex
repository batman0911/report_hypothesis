\chapter{Giới thiệu}
Một kiểm định giả thiết trong mô hình Neyman-Pearson là một tiêu chí quyết định 
cho phép chúng ta lựa chọn giữa 2 giả thiết. Trước khi thực hiện test thống kê, 
ta định nghĩa giả thiết \textbf{null} $H_0$, được giả định là đúng. Giả thiết được
so sánh với đối thiết $H_1$. Đối thiết $H_1$ thường được gọi là giả thiết nghiên cứu
vì thường về lí thuyết các tham số được chỉ định trong giải thiết thay thế này. \\

Các giả thiết có 1 miền xác định tham số  trong không gian tham số $\Theta$ của 
các tham số $\theta$. Giả thiết null $H_0$ được định nghĩa trong miền 
$[\theta \in \Theta_0]$ và đối thiết $H_1$ được định nghĩa trong miền
$[\theta \in \Theta_1]$ và $\Theta_0 \cup \Theta_1 = \Theta$

\section{Sai lầm loại I và sai lầm loại II}
Khi sử dụng phương pháp kiểm định giả thiết chúng ta luôn có xác suất mắc sai lầm.
Ví dụ chúng ta có thể bác bỏ giả thiết null trong khi giả thiết trên thực tế là đúng
hoặc ngược lại, giả thiết null trên thực tế là sai mà chúng ta lại chấp nhận nó. \\ 
Ta định nghĩa 2 loại sai lầm và sẽ tìm hiểu xác suất mắc phải sai lầm đó trong kiểm định 
giả thiết thống kê.

\begin{center}
    \begin{tabular}{| c | c |}
        \hline
        Sai lầm loại I & Giả thiết null là đúng nhưng bị bác bỏ \\
		Sai lầm loại II & Giả thiết null là sai nhưng không bị bác bỏ \\
        \hline
    \end{tabular}
\end{center}

Cả 2 loại này đều dẫn đến quyết định sai lầm trong kết luận.\\
Xác suất mắc sai lầm loại I còn được gọi là \textbf{mức ý nghĩa}
\begin{equation}
	\begin{split}
		\alpha = P(\text{sai lầm loại I}) = P(\text{bác bỏ } H_0 | H_0 \text{ đúng}) \\
		= P(\text{chấp nhận } H_1 | H_0 \text{ đúng})	
	\end{split}
\end{equation}
Xác suất mắc sai lầm loại II $\beta$:
\begin{equation}
	\begin{split}
		\beta = P(\text{sai lầm loại II}) = P(\text{không bác bỏ } H_0 | H_0 \text{ sai}) \\
	= P(\text{chấp nhận } H_0 | H_1 \text{ đúng})
	\end{split}
\end{equation}

\section{Power function}
Đối thiết $H_1: \theta \in \Theta_1$, hàm power của kiểm định được định nghĩa như sau:
\begin{equation}
	\begin{split}
		Power(\theta) = P(\text{bác bỏ } H_0 | H_0 \text{ sai}) = P(\text{chấp nhận } H_1 | H_1 \text{ đúng}) \\
		= 1 - \beta(\theta)
	\end{split}
\end{equation}
Với $\beta(\theta)$ là xác suất mắc sai lầm loại II với 1 ước lượng $\theta$ cho trước.

\section{Kiểm định đồng nhất tốt nhất}
Trước hết chúng ta cùng xem xét ví dụ sau đây \\
\textbf{Ví dụ: } Cho một tổng thể với phân bố chuẩn $N(\mu, 1)$, thực hiện 1 phép thử. 
Kiểm định giả thiết $H_0: \mu = 1$ với đối thiết $H_1: \mu = 2$. Tìm mức ý nghĩa và power của kiểm định 
trong các miền bác bỏ sau: \\
(a) $(2.036, \infty)$ \\
(b) $(1.100, 1.300) \cup (2.461, \infty)$

(a) $R = (2.036, \infty)$,
$$\alpha = P(X > 2.036 | N(1, 1)) = P(\frac{X-1}{1} > \frac{2.036-1}{1}) = P(Z > 1.036) = 0.150$$
$$\beta = P(X > 2.036 | N(2, 1)) = P(\frac{X-2}{1} > \frac{2.036-2}{1}) = P(Z \leqslant 1.036) = 0.514$$
hàm power của test $1 - \beta = 1 - 0.514 = 0.486$

(b) miền bác bỏ $(1.100, 1.300) \cup (2.461, \infty)$, ta có xác suất mắc sai lầm loại I: 
\begin{align*}
	\begin{split}
		\alpha = P(1.100 < X < 1.300 | N(1, 1)) + P(X > 2.461 | N(1, 1) \\
		= P(0.100 < Z < 0.300) + P(Z > 1.461) = 0.150
	\end{split}	
\end{align*}
xác suất mắc sai lầm loại II:
\begin{align*}
	\begin{split}
		\beta = P(X \leqslant 1.100 \ N(2, 1)) + P(1.300 \leqslant X \leqslant 2.461 | N(2, 1)) \\
		= P(Z \leqslant -0.900) + P(-0.700 \leqslant Z \leqslant 0.461) = 0.620
	\end{split}
\end{align*}
hàm power $1 - \beta = 0.380$
Ta có thể thực hiện tính toán trong $R$ như sau 
\begin{lstlisting}{language=R}
	alpha = pnorm(1.3, 1, 1) - pnorm(1.1, 1, 1) + (1 - pnorm(2.461, 1, 1))
	round(alpha, 3)

	beta = pnorm(1.1, 2, 1) + pnorm(2.416, 2, 1) - pnorm(1.3, 2, 1)
	round(beta, 3)
\end{lstlisting}
Ta nhận thấy rằng với cùng 1 mức ý nghĩa $\alpha = 0.150$ thì hàm power có giá trị khác nhau 
với 2 miền bác bỏ trong phần a và phần b. Nhìn chung, sẽ tồn tại 1 test được coi là "tốt hơn"
(theo nghĩa có giá trị power lớn hơn hay là xác xuất mắc sai lầm loại II nhỏ hơn).
Thế nên chúng ta sẽ cùng đi tìm 1 test "tốt nhất" (\textbf{uniformly most powerful}).

\section{p-value hay mức tới hạn}
p-value hay mức tới hạn được định nghĩa là một xác suất nhỏ nhất mắc sai lầm loại I với 
giả định giả thiết null là đúng.
Cách tính p-value được cho trong bảng sau:
\begin{center}
    \begin{tabular}{| c | c |}
        \hline
        $H_1: \theta < \theta_0$ & $P(T \leqslant t_{obs} | H_0)$ \\
		$H_1: \theta > \theta_0$ & $P(T \geqslant t_{obs} | H_0)$ \\
		$H_1: \theta \neq \theta_0$ & $2min\{P(T \leqslant t_{obs} | H_0), P(T \geqslant t_{obs} | H_0)\}$ \\
        \hline
    \end{tabular}
\end{center}
Lưu ý rằng p-value không phải một giá trị cố định mà nó được tính dựa trên mẫu quan sát và
với điệu kiện giả thiết $H_0$ được coi là đúng.
Một giá trị p-value nhỏ thường có ý nghĩa ủng hộ giả thuyết $H_1$. Cho nên, với 1 mức ý nghĩa
$\alpha$ cho trước, chúng ta sẽ bác bỏ $H_0$ nếu giá trị p-value $< \alpha$

\section{Kiểm định với mức ý nghĩa}
Để giải bài toán kiểm định giả thuyết thống kê, ta theo các bước sau:

\textit{Bước 1: \textbf{Giả thuyết}}:
Thiết lập giả thuyết null, $H_0: \theta = \theta_0$. Xác định đối thiết $H_1$ và với tham số $\theta$,
xác định $\theta < \theta_0$ hay $\theta > \theta_0$ hay $\theta \neq \theta_0$.

\textit{Bước 2: \textbf{Test thống kê}}:
Lựa chọn test thống kê phù hợp với phân phối của mẫu, hoặc test thống kê chuẩn hóa với 
giả định $H_0$ là đúng.

Lựa chọn test thống kê, tham số $\hat{\theta}$ là giá trị kì vọng của tham số ứng với $H_0$.
Ví dụ đối với kiểm định cho trung bình $\mu$ thì $\hat{\theta} = \bar{X}$ hoặc
với kiểm định cho tỉ lệ $\pi$ thì $\hat{\theta} = P$.

Đại lượng thống kê chuẩn hóa:
\begin{equation}
	T = t(X) = \frac{\hat{\theta}(X) - \theta_0}{\sqrt{Var[\hat{\theta}(X)]}}
\end{equation}

\textit{Bước 3: \textbf{Tính toán miền bác bỏ}}:
Ta có thể tính toán miền bác bỏ bằng cách sử dụng mức ý nghĩa $\alpha$ hoặc tính p-value.
Sau đó tính toán giá trị $t(X)$ với giả định $H_0$ đúng. Kí hiệu $t(X) = t_{obs}$

\textit{Bước 4: \textbf{Kết luận về mặt thống kê}}:
Sử dụng miền bác bỏ hoặc p-value để đưa ra kết luận về giả thiết null. Nếu $t_{obs}$
rơi vào miền bác bỏ, ta bác bỏ $H_0$, nếu không, ta không ta không đủ cơ sở bác bỏ $H_0$.
Nếu p-value nhỏ hơn mức ý nghĩa $\alpha$, bác bỏ $H_0$, nếu không, ta không ta không đủ cơ sở bác bỏ $H_0$.

\textit{Bước 4: \textbf{Kết luận}}:
Đưa ra kết luận bằng lời.

\textbf{Ước lượng khoảng và kiểm định thống kê}

Khi ước lượng khoảng cho một đại lượng thống kê $\hat{\theta}(X)$ đã biết phân phối
với trung bình $\theta$ và phương sai $\sigma_{\hat{\theta}(X)}$. Tham số chuẩn hóa 
có dạng $\frac{\hat{\theta}(X) - \theta}{\sigma_{\hat{\theta}(X)}}$ (với phân phối đã biết, 
kí hiệu là $T$). Với khoảng tin cậy $(1 - \alpha) * 100\%$, ta có ước lượng khoảng cho $\hat{\theta}(X)$

\begin{equation}
	CI_{1- \alpha}(\theta) = [\hat{\theta}(X) + t_{\alpha/2} \sigma_{\hat{\theta}(X)}, 
	\hat{\theta}(X) + t_{1 - \alpha/2} \sigma_{\hat{\theta}(X)}]
\end{equation}

Trong kiểm định thống kê, khi chuẩn hóa test thống kê có dạng tương đương với dạng chuẩn hóa
của ước lượng khoảng $t_{obs} = \frac{\hat{\theta}(X) - \theta_0}{\sigma_{\hat{\theta}(X)}}$
, ước lượng khoảng và vùng chấp nhận của giả thiết null tuân theo cùng một phân phối thì với 1
độ tin cậy $(1 - \alpha) * 100\%$, giả thiết null cho $\theta$ là $H_0: \theta = \theta_0$ không bị bác bỏ.

Ta có tổng quát qua bảng:
\begin{center}
	\begin{tabular}{| c | c | c |}
		\hline
		Đối thiết & Miền chấp nhận $H_0$ & Ước lượng khoảng $(1 - \alpha)$ \\
		\hline
		$H_1: \theta < \theta_0$ & $t_{obs} \geqslant t_\alpha$ & $(-\infty, \hat{\theta}(X) - t_\alpha \sigma_{\hat{\theta}(X)}]$ \\
		\hline
		$H_1: \theta > \theta_0$ & $t_{obs} \leqslant t_\alpha$ & $[\hat{\theta}(X) - t_\alpha \sigma_{\hat{\theta}(X)}, \infty)$ \\
		\hline
		$H_1: \theta \neq \theta_0$ & $t_{\alpha/2} \leqslant t_{obs} \leqslant t_{1 - \alpha/2}$ &
		$[\hat{\theta}(X) + t_{\alpha/2} \sigma_{\hat{\theta}(X)}, \hat{\theta}(X) + t_{1 - \alpha/2} \sigma_{\hat{\theta}(X)}]$ \\
		\hline
	\end{tabular}
\end{center}