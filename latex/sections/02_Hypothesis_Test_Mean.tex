\chapter{Kiểm định giả thiết cho trung bình}

\section{Kiểm định giả thiết cho trung bình với mẫu có phân bố chuẩn và phương sai đã biết}
Giả thiết null $H_0: \mu = \mu_0$ \\
Thực hiện test với mẫu ngẫu nhiên kích thước $n$, phân bố của $\bar{X}$ là phân bố chuẩn $N(\mu_0, \sigma / \sqrt{n})$
(Giải sử $H_0$ là đúng). \\
Định lý giới hạn trung tâm cũng chỉ ra rằng $\bar{X}$ có phân bố chuẩn với cỡ mẫu lớn. \\
Tham số chuẩn hóa 

\begin{equation}
    Z = \frac{\bar{X} - \mu_0}{\sigma / \sqrt{n}} \sim N(0, 1)
\end{equation}

Ta có

\begin{center}
    \begin{tabular}{| c | c | c | c |}
        \hline
        Đối thiết & $H_1: \mu < \mu_0$ & $H_1: \mu > \mu_0$ & $H_1: \mu \neq \mu_0$ \\
        \hline
        Miền bác bỏ & $z < z_{\alpha}$ & $z > z_{1-\alpha}$ & $\lvert z \rvert > z_{1 - \alpha / 2}$ \\
        \hline
    \end{tabular}
\end{center}

Trong đó $\Phi(z_\alpha) = \alpha$ \\
Nhắc lại, trong ngôn ngữ \lstinline{R}
\begin{lstlisting}{language=R}
    z.x = qnorm(x)
\end{lstlisting}

demo code: 
\begin{lstlisting}{language=R}
    sigma <- 6
    mu <- 40
    pnorm(-2)
    pnorm(2)
\end{lstlisting}


\section{Kiểm định cho trung bình với mẫu có phân bố chuẩn và phương sai chưa biết}

Giả thiết null $H_0: \mu = \mu_0$ \\
Tham số chuẩn hóa được xem xét trong trường hợp này khác với trường hợp đã biết phương sai của tổng thể
:
\begin{equation}
    T = \frac{\bar{X} - \mu_0}{S/ \sqrt{n}} \sim t_{n - 1}    
\end{equation}


Ta có:

\begin{center}
    \begin{tabular}{| c | c | c | c |}
        \hline
        Đối thiết & $H_1: \mu < \mu_0$ & $H_1: \mu > \mu_0$ & $H_1: \mu \neq \mu_0$ \\
        \hline
        Miền bác bỏ & $t < t_{\alpha; n - 1}$ & $t > t_{1-\alpha; n - 1}$ & $\lvert t \rvert > t_{1 - \alpha / 2; n - 1}$ \\
        \hline
    \end{tabular}
\end{center}

Để tính $t_{x; y}$ trong \lstinline{R} ta dùng hàm:

\begin{lstlisting}{language=R}
    t.x.y = qt(x, y)
\end{lstlisting}

\section{Bài tập}
\textbf{pr3}: Mức ý nghĩa $5\%$, tính power với giả thiết $H_0: \mu = 100$ với đối 
thiết $H_1: \mu \neq 100$ với cỡ mẫu 36 có phân phối chuẩn $N(120, 50)$

Tham số chuẩn hóa: 
$$Z = \frac{\bar{X} - \mu}{\sigma / \sqrt{n}}$$
Hàm power: 
$$1 - \beta = P(|Z| > Z_{1 - \alpha / 2}) = P(Z < Z_{\alpha / 2}) + P(Z > Z_{1 - \alpha/2})$$

Code \lstinline{R}:
\begin{lstlisting}
    mu = 100
    mu0 = 120
    sigmaX = 50 / 6
    alpha = 0.05
    z1 = qnorm(alpha / 2, mu, sigmaX)
    z2 = qnorm(1 - alpha / 2, mu, sigmaX)
    pnorm(z1, mu0, sigmaX) + 1 - pnorm(z2, mu0, sigmaX)
\end{lstlisting}

Kết quả $1- \beta = 0.67$
\begin{lstlisting}
    [1] 0.670051
\end{lstlisting}

\textbf{pr4}: Kiểm định cho giả thiết $H_0: \mu = 170$ và đối thiết $H_1: \mu < 170$
cho đại lượng loss trong tập dữ liệu Rubber.

Cách làm đơn giản với code \lstinline{R}:
\begin{lstlisting}
    library(MASS)
    data("Rubber")
    loss <- Rubber$loss
    t.test(loss, alternative = c("less"), mu = 170)
\end{lstlisting}
Kết quả:
\begin{lstlisting}
        One Sample t-test

    data:  loss
    t = 0.33785, df = 29, p-value = 0.631
    alternative hypothesis: true mean is less than 170
    95 percent confidence interval:
        -Inf 202.7589
    sample estimates:
    mean of x 
    175.4333 
\end{lstlisting}
Kết luận: với độ tin cậy $95\%$ ta có thể bác bỏ giả thiết $H_0$, chấp nhận $H_1: \mu < 170$

\textbf{pr12}: Xem xét data frame Fertilize, chiều cao của cây được cho qua biến self

(a) Kiểm định giả thiết cho chiều cao trung bình của cây lớn hơn 17 inchs với mức ý nghĩa $\alpha = 5\%$

(b) Ước lượng khoảng một phía với độ tin cậy $95\%$ cho chiều cao trung bình ($H_1: \mu > 17$)

(c) Tính cỡ mẫu cần thiết để có giá trị power là 0.9 với $\mu_1 = 18$ inchs giả thiết $\sigma = s$

(d) Tính power cho kiểm định trong phần a với $\sigma = s$ và $\mu_1 = 18$

\textit{Lời giải:}

(a) Giả thiết $H_0: \mu = 17$ và đối thiết $H_1: \mu > 17$

\begin{lstlisting}
    library(PASWR)
    data("Fertilize")
    self <- Fertilize$self
    t.test(self, mu = 17, alternative = c("greater"))
\end{lstlisting}

Kết quả thu được:

\begin{lstlisting}
        One Sample t-test

    data:  self
    t = 1.0854, df = 14, p-value = 0.148
    alternative hypothesis: true mean is greater than 17
    95 percent confidence interval:
    16.64196      Inf
    sample estimates:
    mean of x 
    17.575 
\end{lstlisting}
Ta thấy $p-value = 0.148 > \alpha = 0.05$ nên chưa đủ cơ sở để bác bỏ $H_0$.

Kết luận: chưa đủ cơ sở để kết luận chiều cao trung bình $\mu > 17$ inchs với 
mức ý nghĩa $\alpha = 5\%$.

(b) Ước lượng khoảng với độ tin cậy $95\%$ cho chiều cao trung bình của cây là $\mu > 16.642$ inchs

% (c) Với hàm power, ta có 
% Tham số chuẩn hóa $Z = \frac{\bar{X} - \mu_0}{\sigma / \sqrt{n}}$ với $\mu_{obs} = \bar{X}$
% \begin{align*}
%     \begin{split}
%         0.9 = 1 - \beta(\theta) = P(\text{chấp nhận } H_1 | H_1 \text{ đúng}) \\
%         = P(Z > Z_{1 - \alpha}) = P\left(\frac{\bar{X} - \mu_0}{\sigma / \sqrt{n}} > Z_{1 - \alpha}\right)
%         = P\left(\frac{\bar{X} - \mu_0}{s / \sqrt{n}} > Z_{1 - \alpha}\right)
%     \end{split}
% \end{align*}
% Ở đây, ta giả sử $\sigma = s$, kết hớp với phương trình trên ta có:

% $$n > \left(\frac{Z_{0.9}s}{\bar{X} - \mu}\right)^2$$

% \begin{lstlisting}
%     xmean <- mean(self)
%     s2 <- var(self)
%     z_a <- qnorm(0.9)
%     n_min <- (z_a * sqrt(s2) / (xmean - 18))^2
% \end{lstlisting}

% Ta thu được $n_{min} = 38.2747$. Như vậy ta có thể kết luận cỡ mẫu nhỏ nhất để test có power $0.9$, $\mu = 18$ và $\sigma = s$ là $39$

(d) 
\begin{align*}
    \begin{split}
        power(\mu_1) = P(\text{chấp nhận } H_1 | H_1 \text{ đúng}) = P\left(\mu_{obs} > \mu_0 | N\left(\mu_1, \frac{s}{\sqrt{n}}\right)\right) \\
        = P\left( Z > Z_{1 - \alpha} \right) = 0.596
    \end{split}
\end{align*}

\begin{lstlisting}
    n <- length(self)
    z <- qnorm(0.95, 17, sqrt(s2/n))
    round(1 - pnorm(z, 18, sqrt(s2/n)), 3) # = 0.596
\end{lstlisting}

(c)
Thử cho giá trị $n$ khác nhau vào đoạn code \lstinline{R} trên và tính giá trị power ta thấy giá trị nhỏ
nhất để có $power = 0.9$ là $n = 36$

\textbf{pr18}: Tìm power của giả thiết $H_0: \mu = 65$, đối thiết $H_1: \mu > 65$
nếu $\mu_1 = 70$, mức ý nghĩa $\alpha = 0.01$ cho data \lstinline{hard}, data frame \lstinline{Rubber}.
Gỉả sử $\sigma = s$

\textit{Lời giải}:
\begin{align*}
    \begin{split}
        power(\mu_1) = P(\text{chấp nhận } H_1 | H_1 \text{ đúng}) = P\left(\mu_{obs} > \mu_0 | N\left(\mu_1, \frac{s}{\sqrt{n}}\right)\right) \\
        = P(Z > Z_{1 - \alpha}) = 0.496
        % = P\left( Z > \frac{\mu_0 - \mu_1}{s/\sqrt{n}} \right) = 0.988
    \end{split}
\end{align*}

\lstinline{R} code:

\begin{lstlisting}
    library(MASS); data("Rubber")
    hard <- Rubber$hard
    n <- length(hard)
    s2 <- var(hard)
    z <- qnorm(0.99, 65, sqrt(s2/n))
    round(1 - pnorm(z, 70, sqrt(s2/n)), 3) # = 0.469
\end{lstlisting}

