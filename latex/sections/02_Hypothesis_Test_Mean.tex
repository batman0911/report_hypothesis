\chapter{Kiểm định giả thiết cho trung bình}

\section{Kiểm định giả thiết cho trung bình với mẫu có phân bố chuẩn và phương sai đã biết}
Giả thiết null $H_0: \mu = \mu_0$ \\
Thực hiện test với mẫu ngẫu nhiên kích thước $n$, phân bố của $\bar{X}$ là phân bố chuẩn $N(\mu_0, \sigma / \sqrt{n})$
(Giải sử $H_0$ là đúng). \\
Định lý giới hạn trung tâm cũng chỉ ra rằng $\bar{X}$ có phân bố chuẩn với cỡ mẫu lớn. \\
Tham số chuẩn hóa 

$$Z = \frac{\bar{X} - \mu_0}{\sigma / \sqrt{n}} \sim N(0, 1)$$

Ta có

\begin{center}
    \begin{tabular}{| c | c | c | c |}
        \hline
        Đối thiết & $H_1: \mu < \mu_0$ & $H_1: \mu > \mu_0$ & $H_1: \mu \neq \mu_0$ \\
        Miền bác bỏ & $z < z_{\alpha}$ & $z > z_{1-\alpha}$ & $\lvert z \rvert > z_{1 - \alpha / 2}$ \\
        \hline
    \end{tabular}
\end{center}

Trong đó $\Phi(z_\alpha) = \alpha$ \\
Nhắc lại, trong ngôn ngữ $R$
\begin{lstlisting}{language=R}
    z.x = qnorm(x)
\end{lstlisting}

demo code: 
\begin{lstlisting}{language=R}
    sigma <- 6
    mu <- 40
    pnorm(-2)
    pnorm(2)
\end{lstlisting}


\section{Kiểm định cho trung bình với mẫu có phân bố chuẩn và phương sai chưa biết}

Giả thiết null $H_0: \mu = \mu_0$ \\
Tham số chuẩn hóa được xem xét trong trường hợp này khác với trường hợp đã biết phương sai của tổng thể
:
$$T = \frac{\bar{X} - \mu_0}{S/ \sqrt{n}} \sim t_{n - 1}$$

Ta có:

\begin{center}
    \begin{tabular}{| c | c | c | c |}
        \hline
        Đối thiết & $H_1: \mu < \mu_0$ & $H_1: \mu > \mu_0$ & $H_1: \mu \neq \mu_0$ \\
        Miền bác bỏ & $t < t_{\alpha; n - 1}$ & $t > t_{1-\alpha; n - 1}$ & $\lvert t \rvert > t_{1 - \alpha / 2; n - 1}$ \\
        \hline
    \end{tabular}
\end{center}

Để tính $t_{x; y}$ trong $R$ ta dùng hàm:

\begin{lstlisting}{language=R}
    t.x.y = qt(x, y)
\end{lstlisting}


