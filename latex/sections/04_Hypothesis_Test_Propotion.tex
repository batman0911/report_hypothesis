\chapter{Kiểm định giả thiết cho tỉ lệ}

\section{Kiểm định cho tỉ lệ với trong test nhị thức}

Ta xem xét biến ngẫu nhiên có phân bố nhị thức $Y \sim Bin(n, \pi)$, giả thiết null
$H_0: \pi = \pi_0$. 
Test thống kê cho $y_{obs} = \text{Số quan sát thành công}$

Ta có 3 đối thiết ứng với công thức p-value tương ứng như sau:

\begin{center}
    \begin{tabular}{| c | c |}
        \hline
        Đối thiết & p-value \\
        \hline
        $H_1: \pi < \pi_0$ & $P(Y \leqslant y_{obs} | H_0) = \sum_{i = 0}^{y_{obs}}\begin{pmatrix}
            n \\ i
        \end{pmatrix} \pi_0^i(1 - \pi_0)^{n - i}$ \\ 
        \hline 
        $H_1: \pi > \pi_0$ & $P(Y \geqslant y_{obs} | H_0) = \sum_{i = y_{obs}}^{n}\begin{pmatrix}
            n \\ i
        \end{pmatrix} \pi_0^i(1 - \pi_0)^{n - i}$ \\
        \hline
        $H_1: \pi \neq \pi_0$ & $\sum_{i = 0}^n I(P(Y = i) \leqslant P(Y = y_{obs}))\begin{pmatrix}
            n \\ i
        \end{pmatrix} \pi_0^i (1 - \pi_0)^{n- i}$ \\
        \hline
    \end{tabular}
\end{center}

\section{Kiểm định cho tỉ lệ trong test nhị thức (xấp xỉ chuẩn)}
Theo tính chất tiệm cận của ước lượng MLE ta có:

\begin{equation}
    P = \frac{Y}{n} \sim N\left( \pi, \sqrt{\frac{\pi(1 - \pi)}{n}} \right) \text{với } n \to \infty
\end{equation}

Trong thực tế Với $n\pi$ và $n(1 - \pi)$ đều lớn hơn hoặc bằng 10, ta có thể sử dụng xấp xỉ trên.

Đại lượng chuẩn hóa:
\begin{equation}
    Z = \frac{P - \pi_0}{\sqrt{\frac{\pi_0(1 - \pi_0)}{n}}} \sim N(0, 1)
\end{equation}

Khi $|p - \pi_0| > \frac{1}{2n}$ nhiều nhà thống kê cho rằng cần phải thêm vào 1 hiệu chỉnh. 
Hàm \lstinline{prop.test()} trong \lstinline{R} mặc định thêm vào hiệu chỉnh $\pm \frac{1}{2n}$.

Ta có bảng tổng quan như sau

Giải thiết null $H_0: \pi = \pi_0$

\begin{center}
    \begin{tabular}{| c | c |}
        \hline
        Đối thiết & miền bác bỏ \\
        \hline
        $H_1: \pi < \pi_0$ & $z_{obs} < z_\alpha$ \\ 
        \hline 
        $H_1: \pi > \pi_0$ & $z_{obs} > z_{1 - \alpha}$ \\
        \hline
        $H_1: \pi \neq \pi_0$ & $|z_{obs}| > z_{1 - \alpha / 2}$ \\
        \hline
    \end{tabular}
\end{center}

Đối với $|p - \pi_0| > \frac{1}{2n}$ ta có bảng hiệu chỉnh như sau

\begin{center}
    \begin{tabular}{| c | c | c |}
        \hline
        Điều kiện & Hiểu chỉnh & Tham số chuẩn hóa\\
        \hline
        $p - \pi_0 > 0$ & $-\frac{1}{2n}$ & $z_{obs} = \frac{p - \pi_0 - 
        \frac{1}{2n}}{\sqrt{\frac{\pi_0(1 - \pi_0)}{n}}}$ \\ 
        \hline 
        $p - \pi_0 < 0$ & $\frac{1}{2n}$ & $z_{obs} = \frac{p - \pi_0 + 
        \frac{1}{2n}}{\sqrt{\frac{\pi_0(1 - \pi_0)}{n}}}$ \\ 
        \hline
    \end{tabular}
\end{center}

\section{Bài tập}

\textbf{pr17}: Theo Pamplona, Tây Ban Nha, $0.4\%$ người nhập cư năm 2002 đến từ Bolivia.
Tháng 6 năm 2005, một nhóm gồm 3740 người nước ngoài được điều tra ngẫu nhiên thì có 87 người 
Bolivia. Có thể phòng đoán số người nhập cư Boliva đã tăng lên hay không với mức ý nghĩa $\alpha = 0.05$.

Giả thiết null: $H_0: P = \pi_0 = 0.004$, đối thiết $P > \pi_0 = 0.004$
Với $n = 3740$, ta có $n\pi_0$ và $n(1 - \pi_0)$ đều lớn hơn 10 nên ta có thể sử dụng xấp xỉ chuẩn 
để kiểm định cho tỉ lệ trong bài toán này.

Ta có $P = 87/3740$ nên $P - \pi_0 = 0.01926203 > 1/2n = 0.0001336898$, do đó ta phải sử dụng hiệu chỉnh $1/2n$

tham số chuẩn hóa 
$$Z = \frac{P - \pi_0 - \frac{1}{2n}}{\sqrt{\frac{\pi_0(1 - \pi_0)}{n}}} = 18.53333 > 1.644854 = Z_{0.95}$$

Do đó ta có thể kết bác bỏ giả thiết $H_0$

Kết luận: với mức ý nghĩa $5\%$ ta có thể nói rằng tỉ lệ người nhập cư vào Tây Ban Nha đến từ Bolivia (điều tra tháng 6 năm 2005)
đã tăng lên so với thời điểm năm 2002.

Ngoài ra, chúng ta có thể sử dụng phương pháp tính giá trị \lstinline{p-value} được đề cập trong phần 4.1.
Hoặc có thể sử dụng test dựng sẵn trong \lstinline{R} đơn giản như sau:

\begin{lstlisting}
    binom.test(x = 87, n = 3740, p = 0.004, alternative = "greater")
\end{lstlisting}

Kết quả thu được:

\begin{lstlisting}
        Exact binomial test

    data:  87 and 3740
    number of successes = 87, number of trials = 3740, p-value < 2.2e-16
    alternative hypothesis: true probability of success is greater than 0.004
    95 percent confidence interval:
    0.01935316 1.00000000
    sample estimates:
    probability of success 
                0.02326203 
\end{lstlisting}

$\text{\lstinline{p-value}} < \alpha = 0.05$, ta có kết luận tương tự như trên.

\textbf{pr19}: Giám đóc nhà đô thị ở Vitoria, Tây Ban Nha nói rằng ít nhất $50\%$
các căn hộ có nhiều hơn 1 nhà tắm và ít nhất $70\%$ các căn hộ có một thang máy.

(a) Kiểm chứng tuyên bố của giám đốc. Test thống kê với mức ý nghĩa $\alpha = 0.1$. 
Dữ liệu trong cột \lstinline{toilets}, data frame \lstinline{vit2005}.

(b) Kiểm chứng tuyên bố về thang máy với mức ý nghĩa $\alpha = 0.1$ sử dụng phương pháp xấp xỉ
và phương pháp chính xác.

(c) Kiểm tra xem tỷ lệ các căn hộ được xây dựng trước năm 1980 không có nhà để xe có
tỷ lệ có thang máy cao hơn không có thang máy?

\textit{Lời giải}:
(a) 
Bài toán kiểm định: Giả thiết $H_0: \pi = 0.5$ với đối thiết $H_1: \pi > 0.5$

Ta sử dụng hàm dựng sẵn trong \lstinline{R} \lstinline{binom.test()} như sau:

\begin{lstlisting}
    data("vit2005")
    toilets <- vit2005$toilets
    n <- length(toilets)
    ns <- length(toilets[toilets > 1])
    binom.test(ns, n, 0.5, alternative = c("less"), conf.level = 0.1)
\end{lstlisting}

Thu được kết quả:

\begin{lstlisting}
        Exact binomial test

    data:  ns and n
    number of successes = 102, number of trials = 218, p-value = 0.8452
    alternative hypothesis: true probability of success is greater than 0.5
    10 percent confidence interval:
    0.5089864 1.0000000
    sample estimates:
    probability of success 
                0.4678899 
\end{lstlisting}
Ta thấy giá trị $\text{\lstinline{p-value}} = 0.8452 > 0.1$ nên chưa đủ cơ sở để bác bỏ $H_0$

Kết luận: Chưa đủ cơ sở để bác bỏ tuyên bố có ít nhất $50\%$ số căn hộ có nhiều hơn 1 nhà tắm
với mức ý nghĩa $\alpha = 0.1$.

(b) 
Bài toán kiểm định: Giả thiết: $H_0: p = 0.75$ với đối thiết $H_1: p > 0.75$

\textit{Phương pháp xấp xỉ chuẩn}

Đầu tiên ta khởi tạo data cần tính toán:

\begin{lstlisting}
    data("vit2005")
    elevator <- vit2005$elevator
    n <- length(elevator) # = 218
    ns <- length(elevator[elevator == 1]) # = 174
\end{lstlisting}

Kiểm tra hiệu chỉnh:

\begin{lstlisting}
    pi0 <- 0.75
    p <- ns / n # = 174/218 > 0.75
    p - pi0 - 1/(2*n) # > 0
\end{lstlisting}

Ta có tham số chuẩn hóa 
$$Z = \frac{p - \pi_0 - \frac{1}{2n}}{\sqrt{\frac{\pi_0(1 - \pi_0)}{n}}} \sim N(0, 1)$$

Thay số vào phương trình trên ta tính được giá trị của $Z = 1.564124 > Z_{0.9} = 1.281552$
nên có thể bác bỏ giả thiết $H_0$

Kết luận: Chấp nhận khẳng định có ít nhất $75\%$ số căn hộ có thang máy với mức ý nghĩa $\alpha = 0.1$.

\textit{Phương pháp chính xác}

Tính giá trị \lstinline{p-value}:

\begin{align*}
    \text{p-value} = P(Y \geqslant y_{obs} | H_0) = \sum_{i = y_{obs}}^{n}\begin{pmatrix}
        n \\ i
    \end{pmatrix} \pi_0^i(1 - \pi_0)^{n - i} \\
    = \sum_{i = 174}^{218} \begin{pmatrix}
        218 \\ i
    \end{pmatrix} 0.75^i (0.25)^{218-i}
\end{align*}
Ta có thể tính trong \lstinline{R} như sau:

\begin{lstlisting}
    pvalue <- sum(dbinom(ns:n, n, 0.75)) # = 0.05643458
\end{lstlisting}

Ta có $\text{\lstinline{p-value}} < 0.1$ nên có thể bác bỏ $H_0$.

Ngoài ra cũng có thể dùng hàm \lstinline{binom.test()} như sau:

\begin{lstlisting}
    binom.test(x = ns, n = n, p = 0.75, alternative = c("greater"), conf.level = 0.1)
\end{lstlisting}

Ta thu được kết quả tương tự:

\begin{lstlisting}
        Exact binomial test

    data:  ns and n
    number of successes = 174, number of trials = 218, p-value = 0.05643
    alternative hypothesis: true probability of success is greater than 0.75
    10 percent confidence interval:
    0.8288551 1.0000000
    sample estimates:
    probability of success 
                0.7981651 
\end{lstlisting}

Kết luận: Chấp nhận khẳng định có ít nhất $75\%$ số căn hộ có thang máy với mức ý nghĩa $\alpha = 0.1$.